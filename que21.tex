\section{(21) Алгоритм Шоннинга для 3-SAT, использующий случайное блуждание (Осипов Д.)}

Условие задачи все еще \hyperlink{3sat}{\texttt{в том билете}}.

Мы предъявим вероятностное решение \textit{с односторонней ограниченной вероятностью ошибки} (такое было \hyperlink{Freivalds}{\texttt{здесь}}).

\statement{Вероятностное решение (Sch\"oning, 1999), время $O(n^2(4/3)^n)$, шанс ошибки $\leq 1/2$ }{}

Алгоритм описывается даже проще, чем предыдущие. В начале мы берем случайный $x \in \{0, 1\}^n$. Повторим не более $n$ раз следующее: если $x$ не выполняет формулу, то возьмем в ней случайный ложный конъюнкт, случайно выберем \textbf{одну} переменную в нем и изменим ее значение. Описание алгоритма закончено.

\underline{Оценим снизу вероятность того, что этот алгоритм найдет выполняющий набор} $x^*$. Не уменьшая общности, предположим, что выполняющий набор $x^*$ существует и единственный. Заметим тогда (по \hyperlink{flip20}{\texttt{рассуждению из билета 20}}), что при каждой итерации цикла $x$ становится ближе к $x^*$ с вероятностью $\geq 1/3$ и дальше от $x^*$ с вероятностью $\leq 2/3$. Поэтому, не уменьшая общности, еще предположим, что вероятности приближения и отдаления -- \textbf{ровно} $1/3$ и $2/3$ соответственно.

Итак, вероятность того, что $x$ совпадет с $x^*$, моделируется следующей задачей на случайное блуждание по отрезку $[0, N]$. $x$ начинает свой путь в некоторой точке этого отрезка, делает шаг влево с вероятностью $1/3$, вправо -- с $2/3$ (и все время остается в отрезке $[0, N]$), и необходимо оценить вероятность того, что в течение $n$ шагов он когда-нибудь посетит 0.

Не уменьшая общности, для того, чтобы $x$ посетил 0 в течение $n$ шагов, \textbf{достаточно} (конечно, не необходимо) два условия:
\begin{enumerate}
    \item Случайно выбранный в начале алгоритма $x$ оказался $x^*$ на расстоянии $n/3$ от $x^*$
    \item За $n$ шагов из точки $n/3$ он придет в 0, совершив $2n/3$ шагов влево и $n/3$ шагов вправо, не выходя при этом за границу отрезка 0.
\end{enumerate}

Сейчас мы посчитаем вероятности этих двух событий, их произведение и будет оценкой снизу на вероятность того, что алгоритм найдет выполняющий набор.

Вероятность первого события равна $P_1 = \frac{{n\choose{n/3}}}{2^n}$ так как из $2^n$ равновероятных наборов $\in \{0, 1\}^n$ мы должны выбрать тот, у которого ровно $n/3$ позиций, в которых он и $x^*$ различаются. 

Для подсчета вероятности второго события воспользуемся следующей задачей.

\statement{Теорема (задача о пьянице и канаве).}{Сколько существует путей из точки $S=P-Q>0$ в точку $0$, состоящих ровно из $P$ шагов влево, $Q$ шагов вправо и не выходящих за точку 0? Ответ: $\frac{P-Q}{P+Q} {P+Q\choose P}$.}

Доказательство задачи можно найти \href{https://drive.google.com/file/d/1KaZ5K6OAp6HiJGrxcBXQsNTa5yug4_FE/view?usp=sharing}{\texttt{здесь}}. \qed

В нашем случае количество шагов влево $P = 2n/3$, вправо $Q=n/3$, так что всего таких путей $\frac{1}{3}{n\choose n/3}$. Для фиксированного пути с $P$ шагами влево и $Q$ шагами вправо вероятность, что $x$ пройдет именно его, равна $(1/3)^P (2/3)^Q = (1/3)^{2n/3} (2/3)^{n/3}$, так что:

$$P_2 = \frac{1}{3} {n\choose n/3} (1/3)^{2n/3} (2/3)^{n/3}$$

\newcommand{\scm}{\overset{\text{c}}{\sim}}
Символом $\scm$ будем обозначать <<эквивалентность с точностью до константы>>, т.е: $$[f \scm g] \iff [\exists C > 0: \; f \sim Cg]$$

С помощью формулы Стирлинга $n! \sim \sqrt{2\pi n}\left(\frac{n}{e}\right)^n \scm \sqrt{n}\left(\frac{n}{e}\right)^n$ можно убедиться, что:

$$ P_1 \scm \frac{1}{\sqrt n}\left(\frac{3}{2^{5/3}}\right)^n$$
$$ P_2 \scm \frac{1}{\sqrt n}\left(\frac{1}{2^{1/3}}\right)^n$$

И поэтому вероятность успеха асимптотически хотя бы 
$$P \geq P_1 \cdot P_2 \scm \frac{1}{\sqrt n}\left(\frac{3}{2^{5/3}}\right)^n \cdot \frac{1}{\sqrt n}\left(\frac{1}{2^{1/3}}\right)^n = \frac{1}{n}\left(\frac{3}{4}\right)^n$$

Однако этот алгоритм работает за $O(n)$ времени! Его можно повторить много раз, увеличивая шансы на успех. В частности, если повторить его $n\left(\frac{4}{3}\right)^n = L$ раз, то имеем вероятность неудачи:

$$\left(1-\frac{1}{L}\right)^L \leq \frac{1}{e} \leq \frac{1}{2}$$

Что и приводит нас к требуемому результату. \qed

\statement{NB: }{А если повторить в $q$ раз больше, то есть $qn\left(\frac{4}{3}\right)^n$ раз, то вероятность неудачи $\leq\left(\frac{1}{2}\right)^q$ можно выбрать сколь нужно малой.}











