\section{Приближенный алгоритм для задачи о рюкзаке} 
Напомним сначала классическое, точное решение задачи о рюкзаке методом динамического программирования.

\statement{Задача (о рюкзаке с повторениями).}{} Пусть есть $n$ видов вещей, $i-$тая вещь имеет вес $w_i$ и стоимость $v_i$. Количество каждого вида вещей неограничено. Пусть $W$ -- максимальный вес, который выдерживает рюкзак. Найти максимальную стоимость по всем наборам вещей, суммарный вес которого не превышает $W$.

\statement{Точное решение за $O(n\sum{v_i}=nV)$}{}. Динамическое программирование. Пусть $K[v]$ -- минимальный вес набора стоимостью ровно $v$. Тогда $K[0] = 0$, $K[v] = \underset{i=0}{\overset{n}{\min}} \{K[v-v_i] + w_i\}$. Ответ на задачу: $\max\{v : K[v] \leq W\}$. 

Таким образом заполняется массив длины $V+1$, на каждый поиск минимума уходит $O(n)$ времени, на поиск ответа $O(n)$, всего $O(nV)$.  $\blacksquare$

Теперь рассмотрим решение, которое может выдавать решение с заданной точностью. Именно, для всякого $0 < \e < 1$ это решение будет работать за $O(\frac{n^3}{\e})$ времени, а ценность найденного набора будет отличаться от оптимальной на множитель, не превышающий $(1-\e)$.

\statement{Приближенное $\frac{1}{1-\e}$-оптимальное решение за $O(\frac{n^3}{\e})$.}{} 

Зафиксируем параметр $\e>0$. Заменим все $v_i$ на: $$\hat{v_i} = [\frac{n}{\e} \cdot \frac{v_i}{v_{max}}]$$ Запустим на новом наборе алгоритм ДП выше. Описание алгоритма закончено.

Оценим время работы. $V = \sum v_i \leq n\cdot \frac{n}{\e} = \frac{n^2}{\e}$. Поэтому время $O(n \cdot \frac{n^2}{\e}) = O(\frac{n^3}{\e})$.

Теперь точность. 

Пусть оптимальное решение исходной задачи -- набор $S$, его стоимость с точки зрения старой задачи $K^* = \summ{i\in S} v_i$. 

С точки зрения новой задачи сумма этого набора оценивается как: 
$$\summ{i\in S}\hat{v_i} = \summ{i\in S} [\frac{v_i n}{\e v_{max}}] \geq \summ{i\in S} (v_i \cdot \frac{n}{\e v_{max}} - 1) \geq K^*\frac{n}{\e v_{max}} - n$$

С точки зрения новой задачи набор $S$ необязательно оптимален. То есть, если $\hat{S}$ -- оптимальный с точки зрения новой задачи набор, то имеем $$\summ{i\in\hat S}\hat{v_i} \geq \summ{i\in S} \hat{v_i} \geq K^*\frac{n}{\e v_{max}} - n$$

Нужно оценить, насколько стоимости наборов $S$ и $\hat S$ отличаются с точки зрения старой задачи, т.е. сравнить величины $\summ{i\in S}v_i = K^*$ и $\summ{i\in\hat S}v_i$. Что же, так как $\hat{v_i} \leq \frac{v_i n}{\e v_{max}}$,

$$\summ{i\in\hat S}v_i \geq \summ{i\in\hat S} \hat{v_i} \frac{\e v_{max}}{n} \geq (K^*\frac{n}{\e v_{max}} - n)\frac{\e v_{max}}{n} = K^* - \e v_{max} \geq K^* - \e K^* = K^*(1-\e)$$

Таким образом, полученное решение  хуже оптимального не более, чем в $\frac{1}{1-\e}$ раз. $\blacksquare$