\hypertarget{3sat}{}
\section{(20) Слабоэкспоненциальные детерминированные алгоритмы SAT для 3-КНФ (Осипов Д.)}
\subsection{Начальные сведения}
\begin{problem*}[SAT]
Для данной пропозициональной формулы от $n$ переменных определить, выполнима ли она.
\end{problem*}

\statement{Решение за $O(2^n)$.}{} Переберем все $2^n$ возможных наборов значений переменных. \qed

\statement{Факт.}{Задача SAT NP-трудна: любую задачу из NP можно свести к SAT. Научимся решать SAT за полином $\implies$ научимся решать любую NP-задачу за полином и получим $P=NP$. Докажем, что SAT не решается за полином $\implies$ автоматически $P\neq NP$.}

\statement{Факт.}{SAT сводится к 3-SAT}.

\begin{problem*}[3-SAT]
	Пусть дана пропозициональная формула от $n$ переменных в 3-КНФ (каждый конъюнкт содержит не более трех слагаемых). Определить, выполнима ли она.
\end{problem*}

\subsection{Метод расщепления: $O(1.92^n)$, $O(1.84^n)$}
\statement{Решение за $O\left(\sqrt[3]{7}^n\right) = O(1.92^n)$ (метод расщепления-1).}{}

Рекурсивный алгоритм. Выделим один из конъюнктов $$\ldots\land(x_1^{\sigma_1} \lor x_2^{\sigma_2} \lor x_3^{\sigma_3})\land\ldots$$ Из всех восьми возможных наборов значений $x_1, x_2, x_3$ конкретно под этот конъюнкт подходят только семь~-- все, кроме $(x_1, x_2, x_3) = (\bar{\sigma_1}, \bar{\sigma_2}, \bar{\sigma_3})$.

Для каждого из семи наборов значений делаем следующее: подставляем его в формулу и запускаем алгоритм рекурсивно на получившейся формуле от $n-3$ переменных.

Время работы определяется соотношением $T(n) = 7T(n-3) + O(1)$, откуда немедленно $T(n) = O(7^{n/3})$ \qed.

\needpicture

\statement{Решение за $\sim O(1.84^n)$ (метод расщепления-2)}.

Снова рекурсивный алгоритм. Выделим один из конъюнктов $$\ldots\land(x_1^{\sigma_1} \lor x_2^{\sigma_2} \lor x_3^{\sigma_3})\land\ldots$$ Рекурсивно рассмотрим три случая, когда этот конъюнкт может быть истинен:
\begin{enumerate}
\item либо $x_1 = \sigma_1$,
\item либо $x_1 = \neg\sigma_1$ и $x_2 = \sigma_2$,
\item либо $x_1 = \neg\sigma_1$, $x_2 = \neg\sigma_2$ и $x_3 = \sigma_3$
\end{enumerate}
Для каждого из этих случаев сделаем подстановку и рекурсивно решим подзадачи: для формул от $n-1$, $n-2$ и $n-3$ переменных соответственно.

Время работы описывается соотношением $T(n) = T(n-1) + T(n-2) + T(n-3) + O(1)$. $T(n) = O(1.84^n)$~-- его приближенное решение. \qed

\needpicture

\subsection{Метод локального поиска: $O(1.74^n)$}
Следующее решение основано на методе <<локального поиска>>. Зададим на множестве векторов $\{0, 1\}^n$ метрику $d(x, y) = \text{количество позиций, в которых } x \text{ и } y \text{ различны}$. Для данного вектора $x$ и натурального $r$ определим шар $H(x, r)$~-- множество векторов, отличающихся от $x$ не более, чем в $r$ позициях.

Нам понадобится следующая \hypertarget{flip20}{\textit{вспомогательная задача}}.

\statement{Вспомогательная задача.}{Дан вектор $x \in \{0, 1\}^n$ и натуральный радиус $r$. Проверить, есть ли в шаре $H(x, r)$ выполняющий набор для данной 3-КНФ формулы.}

\statement{Решение вспомогательной задачи за $O(3^r)$.}{} Рекурсивный алгоритм. Сначала проверим формулу на наборе $x$. Если в нем формула не выполнена, выделим в ней любой ложный конъюнкт $(x_a^{\sigma_a} \lor x_b^{\sigma_b} \lor x_c^{\sigma_c})$. Если в $H(x, r)$ присутствует выполняющий набор $x^*$, то $x^*$ \textbf{не} совпадает с $x$ хотя бы в одной из позиций $a,b,c$. Рассмотрим три набора $x^{(a)}, x^{(b)}, x^{(c)}$, каждый из которых получается из $x$ инвертированием $a$-той, $b$-той и $c$-той переменной соответственно. Хотя бы один из наборов  $x^{(a)}, x^{(b)}, x^{(c)}$ будет на единицу ближе к $x^*$ (ведь изменилась всего одна переменная). Запустим от каждого из них этот алгоритм рекурсивно. Тогда на глубине рекурсии, не превосходящей $r$, набор $x^*$ найдется, если он есть в $H(x, r)$. Очевидно, решение работает за $O(3^r)$. \qed

Теперь мы готовы решать нашу задачу $3-SAT$.

\statement{Решение за $O\left(\sqrt{3}^n\right) = O(1.74^n)$ (локальный поиск).}{}

Обозначим $\mathbf 0 = (0, \ldots, 0)$ и $\mathbf 1 = (1, \ldots, 1)$~-- вектора в $\{0, 1\}^n$. Заметим, что всё пространство $\{0, 1\}^n$ покрывается двумя шарами $H(\mathbf 0, n/2)$ и $H(\mathbf 1, n/2)$. Действительно, каждый вектор длины $n$ имеет либо хотя бы $n/2$ единиц, либо хотя бы $n/2$ нулей, откуда следует требуемое. Значит, достаточно за $O(3^{n/2})$ поискать выполняющий набор в каждом из двух шаров. Итоговое время работы $O(3^{n/2}) + O(3^{n/2}) = O(3^{n/2})$. \qed




