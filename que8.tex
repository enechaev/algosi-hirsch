\let\bf\bfseries
\let\it\itshape
\section{(WIP)Алгоритм Эдмондса-Карпа}\label{edmonds_karp}
Вариант реализации алгоритма Форда-Фалкерсона, где в качестве алгоритма поиска пути используется поиск в ширину (предполагается, что у всех ребер единичная длина), называется {\bf\it алгоритмом Эдмондса-Карпа}. Для него есть хорошая оценка времени работы $O(|V||E|^2)$. Она хорошая, потому что не зависит от величины максимального потока.

Обозначим как $\delta_f(u,v)$ кратчайшее расстояние между вершинами $u$ и $v$ в остаточной сети $G_f$.
\begin{lemma}
	Для всех вершин $v\in V\smallsetminus\{s,t\}$ длина кратчайшего пути $\delta_f(s,v)$ в остаточной сети $G_f$ монотонно возрастает при выполнении алгоритма.
\end{lemma}
\begin{proof}
	Будем доказывать от обратного. Предположим, что существует такое увеличение потока, которое приводит к уменьшению длины кратчайшего пути из $s$ к некоторой вершине $v$. $f$~-- поток перед этим увеличением по пути, $f'$~-- поток после этого увеличения. Выберем $v$, чтобы $\delta_{f'}(s,v)$ было минимальным. Тогда $\delta_{f'}(s,v)<\delta_f(s,v)$. Пусть $u$~-- вершина перед $v$ в этом пути в $G_{f'}$, то есть $\delta_{f'}(s,u)=\delta_{f'}(s,v)-1$. По выбору $v$ $\delta_{f'}(s,u)\ge\delta_f(s,u)$ (иначе противоречие с минимальностью). 
	
	Теперь предположим, что $(u,v)\in E_f$. Но тогда $$\delta_{f'}(s,v)=\delta_{f'}(s,u)+1\ge\delta_f(s,u)+1=\delta_f(u,v)$$
	Если же $(u,v)\not\in E_f$ (но $(u,v)\in E_{f'}$), то после увеличения вырастет поток на ребре $(v,u)$. Заметим, что тогда в 
\end{proof}

