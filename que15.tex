\hypertarget{Проверка равенства полиномов}{\section{Проверка равенства полиномов. Лемма Шварца-Циппеля.}}

\statement{Лемма Шварца-Зиппеля}{$0\ne p\in\mathbb{Z}[x_1,...,x_m], deg x_i\le d, A\subset\mathbb{Z}, |A|<\infty,$ тогда $|\{(i_1,...,i_m)\in A^m|p(i_1,...,i_m)=0\}|\le mdA^{m-1}$}

Докажем индукцией по количеству переменных: при $m=1$, очев, корней $\le d$.
\\Переход: 
Очев, можно написать $p(x_1,...,x_m)=x_m^d*y_d+...+x_m^0*d_0,$ где $\{y_i\}\subset Z[x_1,...,x_{m-1}]$
\\1)Посмотрим на те корни, где $y_d=0$, таких кортежей по предположению индукции $\le (m-1)dA^{m-2}*A$
\\2)Тут $y_d\ne 0$. Каждая комбинация $x_1,...,x_{m-1}$ даст нам многочлен от $x_m$ у которого корней меньше, чем $d$. То есть тут $\le dA^{m-1}$
\\Итого $(m-1)dA^{m-1}+dA^{m-1}=mdA^{m-1}. \blacksquare$ 
\\Теперь давайте проверять равенство $p_1$ и $p_2$(это, очев, то же самое что проверять $p_1-p_2$ на равенство тождественному нулю), сколько в них переменных мы знаем изначально, максимальную степень можно посчитать, теперь берем $p_1-p_2$ и втыкаем в него любые $mdA^{m-1}+1$ наборов, если встретился хотя бы один не ноль, то многочлены не равны, если не встретились, то мы уверены, что многочлены равны(в качестве $A$ можно взять что угодно, но надо учесть, что $A^m$ должна быть $\ge mdA^{m-1}$, т.е. $|A|\ge md$)