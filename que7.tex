\let\bf\bfseries
\let\it\itshape
\section{(WIP)Транспортные сети. Задача о максимальном потоке. Разрез. Теорема о максимальном потоке и минимальном разрезе. Алгоритм Форда-Фалкерсона}
Фактически эта глава~--- просто пересказ параграфов из кормена в правильном порядке.
\begin{definition}
	{\bfseries Транспортной сетью} называется ориентированный граф $G=\langle V,E\rangle$ с функцией $c\colon V\times V\to\mathbb{N}$, которая называется {\bf\it пропускной способностью}, причем $c(u,v)=0\iff (u,v)\not\in E$, а также двумя выделенными вершинами~--- {\bf\it источником} $s$ и {\bf\it стоком} $t$.
\end{definition}
Внимание! В источник могут входить ребра, а из стока выходить.

Для удобства предполагается, что любая вершина находится на некотором пути от источника к стоку (то есть граф связный).
\begin{example}
	\needpicture
\end{example}
\begin{definition}
	{\bfseries Потоком} называется функция $f\colon V\times V\to\mathbb{R}$, для которой выполняются следующие свойства:
	\begin{enumerate}
		\item $\forall(u,v)\in V\times V\colon f(u,v)\le c(u,v)$
		\item $\forall(u,v)\in V\times V\colon f(u,v)=-f(v,u)$
		\item $\forall u\in V\smallsetminus\{s,t\}\colon \sum_{v\in V} f(u,v)=0$
	\end{enumerate}
	{\bfseries Величиной} потока называется число $|f|\overset{\mathrm{def}}{=}\sum_{v\in V}f(s,v)$.
\end{definition}
Одна из возможных интерпретаций этого~-- электрическая цепь. Тогда все свойства потоков превращаются в правила Кирхгофа.

Обратите внимание, что если есть ребро $(u,v)$ и с потоком $f(u,v)\ne0$, но нет ребра $(v,u)$, то тем не менее $f(v,u)=-f(u,v)\ne0$. Подумайте, почему если между вершинами нет ребра ни в каком направлении, то поток между ними нулевой\footnote{потому что $0=c(u,v)\ge f(u,v)=-f(v,u)\ge -c(v,u)=0$}.
\begin{problem}(о максимальном потоке)
	Дана транспортная сеть. Нужно найти в ней поток максимальной величины.
\end{problem}
Базовая идея: взять какой-нибудь (например, тривиальный) поток и увеличивать его, пока можно. Осталось только научиться все это делать.

Чтобы правильно увеличивать членб, нужно еще несколько определений.
\begin{definition}
	Для сети $G$ и потока $f$ {\bf\it остаточной пропускной способностью} ребра $(u,v)$ называется величина $c_f(u,v)=c(u,v)-f(u,v)$. {\bf\it Остаточной сетью} $G_f=\langle V,E_f\rangle$ называется сеть на вершинах графа $G$ с множеством ребер $E_f=\{(u,v)\in V\times V|c_f(u,v)>0\}$ с пропускной способностью $c_f$ и теми же источником и стоком.
\end{definition}
Обратите внимание, что если в $G$ есть ребро $(u,v)$, но нет ребра $(v,u)$ (то есть его пропускная способность 0), то остаточная пропускная способность $c_f(v,u)=c(v,u)-f(v,u)=f(u,v)$, то есть если между вершинами есть одно из ребер с ненулевым потоком, то в остаточную сеть попадут оба.
%Понятно, что если ребро есть, но поток по нему нулевой, то его можно увеличить. Если же ребра нет, то остаточная пропускная способность будет нулевой.
Получается, что $|E_f|\le 2|E|$.

\begin{lemma}\label{someshit1} %(о том, как поток в остаточной сети связан с потоком в исходной)
	Пусть $\langle G,c\rangle$~--- транспортная сеть, $f$~--- поток в ней, $G_f$~-- остаточная сеть и в ней задан поток $f'$. Тогда $f+f'$~--- поток в $G$, а его величина $|f+f'|=|f|+|f'|$.
\end{lemma}
\begin{proof}
	Проверим условия на потоки:
	\begin{enumerate}
		\item $$(f+f')(u,v)=f(u,v)+f'(u,v)\le f(u,v)+(c(u,v)-f(u,v))=c(u,v)$$
		\item $$(f+f')(u,v)=f(u,v)+f'(u,v)=-f(v,u)-f'(v,u)=-(f+f')(v,u)$$
		\item $$\sum_{v\in V}(f+f')(u,v)=\sum_{v\in V}f(u,v)+\sum_{v\in V}f'(u,v)=0$$
	\end{enumerate}
	Поэтому это поток.
	
	С величиной все понятно:
	$$|f+f'|=\sum_{v\in V}(f+f')(s,v)=\sum_{v\in V}f(s,v)+\sum_{v\in V}f'(s,v)=|f|+|f'|$$
\end{proof}
\begin{definition}
	{\bf\it Увеличивающим путем} называется простой путь между $s$ и $t$ в $G_f$.
\end{definition}
\begin{lemma}\label{someshit2}
	$G,c,s,t$~--- сеть с потоком $f$, $p$~--- увеличивающий путь в $G_f$. Определим $f_p\colon V\times V\to\mathbb{R}$.
	$$
	f(u,v)=
	\begin{cases}
		c_f(p), & (u,v)\in p,\\
		-c_f(p), & (v,u)\in p,\\
		0, & \mathrm{otherwise}
	\end{cases}
	$$
	где $c_f(p)=\min\{c_f(u,v)|(u,v)\in p\}$.
	Тогда $f_p$~--- поток в $G$ с величиной $c_f(p)>0$.
\end{lemma}
\begin{proof}
	\begin{enumerate}
		\item $$f(u,v)\le c_f(p)\le c_f(u,v)=c(u,v)-f(u,v)\le c(u,v)$$
		\item \ldots
		\item Заметим, что для любой вершины $v$ (не источника и не стока) в путь входит ровно одно ребро $(u,v)$ и ровно одно ребро $(v,w)$, то есть у всех остальных ребер потоки будут нулевые, а у этих они отличаются знаком, поэтому сумма потоков  $\sum_{v\in V}f_p(u,v)=0$.
	\end{enumerate}
\end{proof}
Из лемм~\ref{someshit1} и \ref{someshit2} следует, что поток на каждом ребре пути может быть увеличен на величину $c_f(p)$ (которая называется {\bf\it пропускной способностью пути}), чтобы не нарушить условия на сумму потоков и ограничение пропускной способности.

Теперь осталось научиться определять, чем максимальный поток отличается от немаксимального. Для этого нужно еще несколько определений.
\begin{definition}
	{\bf\it Разрезом} сети $G$ называется разбиение $V=S\sqcup T$, что $s\in S,t\in T$.
	{\bf\it Чистым потоком} потока $f$ через разрез $(S,T)$ называется $f(S,T)\overset{\mathrm{def}}{=}\sum_{x\in S}\sum_{y\in T}f(x,y)$. 
	{\bf\it Пропускной способностью разреза} называется $c(S,T)\overset{\mathrm{def}}{=}\sum_{x\in S}\sum_{y\in T}c(x,y)$.
	{\bf\it Минимальный разрез}~--- это тот, у которого пропускная способность минимальна.
\end{definition}

\begin{lemma}\label{someshit3}
	Чистый поток через любой разрез равен величине потока.
\end{lemma}
\begin{proof}
	Заметим, что $\sum_{x\in S}\sum_{y\in S}f(x,y)=0$.
	\begin{multline*}
	$$
	\sum_{x\in S}\sum_{y\in T}f(x,y)=\sum_{x\in S}\sum_{y\in V}f(x,y)-\sum_{x\in S}\sum_{y\in S}f(x,y)=\\
	\sum_{x\in S}\sum_{y\in V}f(x,y)=\sum_{y\in V}f(s,y)+\sum_{x\in S\smallsetminus \{s\}}\sum_{y\in V}f(x,y)=\sum_{y\in V}f(s,y)=|f|
	$$
	\end{multline*}
\end{proof}
\begin{lemma}
	Величина любого потока не превышает пропускную способность любого разреза.
\end{lemma}
\begin{proof}
	$$
	|f|=\sum_{x\in S}\sum_{y\in T}f(x,y)\le\sum_{x\in S}\sum_{y\in T}c(x,y)=c(S,T)
	$$
\end{proof}
\begin{theorem} {\bf (О максимальном потоке и минимальном разрезе)}
	$G,c,s,t$~--- транспортная сеть с потоком $f$. Следующие утверждения эквивалентны:
	\begin{enumerate}
		\item $f$~--- максимальный поток в $G$.
		\item Остаточная сеть $G_f$ не содержит увеличивающих путей.
		\item $|f|=c(S,T)$ для некоторого разреза $(S,T)$.
	\end{enumerate}
\end{theorem}
\begin{proof}$ $\newline %dirty hack to newline
	\begin{itemize}
		\item[$\mathrm{1}\Rightarrow\mathrm{2}$] Если есть увеличивающий путь, то по лемме~\ref{someshit2} можно построить поток со строго большей величиной, то есть $f$ не максимальный.
		\item[$\mathrm{2}\Rightarrow\mathrm{3}$] Предположим, что нет увеличивающего пути. Определим $S=\{v\in V|\exists p\colon s\to v\textrm{ in }G_f\}, T=V\smallsetminus S$. Понятно, что это разрез. В нем для любой пары $(u,v)\in S\times T$ выполняется $f(u,v)=c(u,v)$, потому что иначе бы ребро $(u,v)$ попало бы в $E_f$ (напомню, что там находятся только те ребра, у которых неотрицательная остаточная пропускная способность) а значит существовал бы путь из $s$ в $v$, это противоречит $v\in T$. $$|f|=\sum_{x\in S}\sum_{y\in T}f(x,y)=\sum_{x\in S}\sum_{y\in T}c(x,y)$$
		\item[$\mathrm{3}\Rightarrow\mathrm{1}$] Из леммы~\ref{someshit3} следует, что $|f|\le c(S,T)$. Поэтому если достигается равенство, то $f$~--- максимальный.
	\end{itemize}
\end{proof}

Теперь мы умеем все доказывать, чтобы описать алгоритм из базовой идеи, который называется {\bf\it алгоритмом Форда-Фалкерсона}.
\begin{lstlisting}[escapeinside=``]
function FFA(`$\langle G=\langle V,E\rangle,c,s,t\rangle$`):
	foreach `$(u,v)\in E$`:
		`$f(u,v)$` := 0
		`$f(v,u)$` := 0
	while `$\exists p\colon s\to t \textrm{ in } G_f$`:
		`$c_f(p)$` := `$\min\{c_f(u,v)|(u,v)\in p\}$`
		foreach `$(u,v)\in p$`:
			`$f(u,v)$` += `$c_f(p)$`
			`$f(v,u)$` := `$-f(u,v)$`
\end{lstlisting}

На практике, понятно, он возникает в основном только с целыми числами. В этом предположении (в кормене написано, что если значения пропускной способности иррациональные, то ничего работать не будет, я не понимаю, почему) время его работы составляет $O(|E||f^*|)$, где $f^*$~--- максимальный поток.

Первый цикл выполняется за время $\Theta(|E|)$, второй цикл выполняется не более $|f^*|$ раз (потому что величина потока в каждую итерацию увеличивается хотя бы на 1). Как долго он выполняется зависит от того, как мы храним граф. Если его как-то там непонятно хранить я это потом напишу, то поиск в ширину работает за $O(V+E')=O(E)$.