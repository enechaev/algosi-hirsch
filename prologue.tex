\section*{Информация}

Этот проект~-- коллективный \textbf{конспект} по второй части курса <<Математические основы алгоритмов>>, впервые прочитанного первокурсникам МКН СПбГУ в первой половине ii семестра 2020 года Эдуардом Алексеевичем Гиршем.

Актуальные исходники: \url{https://www.overleaf.com/read/hnbkrkyknbpk} и \url{https://github.com/gogochushij/algosi-hirsch}

\section*{Ошибки и корректура}

Это $\beta$-версия конспекта. Цель проекта~-- создать не просто билеты к экзамену, а написать \textbf{максимально понятный}, насколько возможно достаточный для самостоятельного изучения курса текст. В связи с этим,

\begin{center} {\color{red} задавайте ВСЕ ваши вопросы авторам конспекта!}\end{center}

Каждый Ваш вопрос по материалу это не только возможность глубже разобраться в материале, но и обратная связь, направленная на повышение читабельности, \textbf{доступности для понимания} текста. Мы \textbf{непременно} ждем Ваших вопросов. Вы можете задавать вопросы по любым главам любому из троих авторов: Денис Осипов, Иван Ермошин, Егор Нечаев.

\section*{Билеты про онлайн-алгоритмы}

Прямо сейчас в конспекте не хватает двух билетов про онлайн-алгоритмы. Если вы хотите помочь нам и написать хотя бы один, или же \textbf{сообщить об ошибке}, напишите \url{http://vk.com/gogochushij}.

А это просто какая-то рабочая ссылка (\href{https://docs.google.com/spreadsheets/d/17MKhLVzCyYvEKlm6W5Bb-6uUDNyv0QLmdvuh4N6JfXI/edit?usp=sharing}{algosi-authors})

\section*{Обозначения}

Символом \heart~обозначены разделы, которые не входят в экзамен (еще не всё проставлено).

Символом $\bigtriangleup$ обозначаются окончания описаний алгоритмов~-- в противопоставление с символом $\square$, который оканчивает доказательства теорем. 
