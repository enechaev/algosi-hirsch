\section*{Информация}

Этот проект~-- коллективный \textbf{конспект} по второй части курса <<Математические основы алгоритмов>>, впервые прочитанного первокурсникам МКН СПбГУ в первой половине ii семестра 2020 года Эдуардом Алексеевичем Гиршем. Порядок глав идет в соответствии с вопросами к второй части экзамена (главы 19,22,23 так и не написаны).

Цель проекта~--  написать \textbf{максимально понятный}, насколько возможно достаточный для самостоятельного изучения курса текст. В связи с этим материал местами изложен подробнее, чем было проговорено на лекциях.

\textbf{Disclaimer}. Этот конспект написан по предоставленным лектором курса Э.А.~Гиршем книгам, а не по лекциям. Он может претендовать на соответствие лекциям в целом, но допускаются отличия в изложении, не влияющие значительно на понимание материала.

Актуальные исходники: \url{https://www.overleaf.com/read/hnbkrkyknbpk} и \url{https://github.com/gogochushij/algosi-hirsch}

\section*{Обозначения}

Символом \heart~обозначены разделы, которые не входят в экзамен (еще не всё проставлено).

Символом $\bigtriangleup$ обозначаются окончания описаний алгоритмов~-- в противоположность символу $\square$, который оканчивает доказательства теорем. Иногда доказательство некого утверждения об алгоритме удобно приводить параллельно описанию алгоритма~-- в этом случае они включаются в описание и содержатся в том же $\bigtriangleup$-блоке.

\section*{Контакты и донаты}

Сообщить об ошибке~-- Денису Осипову (\texttt{st077024@student.spbu.ru}),\\ Егору Нечаеву (\texttt{nechaev.e.01@gmail.com}), Ивану Ермошину (\texttt{ivan.ermoshin.02@gmail.com}). Задавать вопросы тоже можно. 

\begin{center}
    Мы принимаем \textbf{\href{https://ko-fi.com/triangulation}{донаты}} (делятся поровну между тремя авторами).
\end{center}