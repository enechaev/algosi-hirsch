\documentclass{article}

\usepackage[english, russian]{babel}
\usepackage[utf8]{inputenc}
\usepackage{amsmath,amssymb,amsthm}
\usepackage{parskip}
\usepackage{graphicx}
\usepackage{listings} %нужно заменить листинги на algorithm2e
\usepackage[ruled]{algorithm2e}
\usepackage{comment}
\usepackage{hyperref}
\usepackage{pgffor}

% Margins
\usepackage[top=2.5cm, left=3cm, right=3cm, bottom=3.0cm]{geometry}
% Colour table cells
\usepackage[table]{xcolor}

\usepackage{tikz}

% Get larger line spacing in table
\newcommand{\tablespace}{\\[1.25mm]}
\newcommand\Tstrut{\rule{0pt}{2.6ex}}         % = `top' strut
\newcommand\tstrut{\rule{0pt}{2.0ex}}         % = `top' strut
\newcommand\Bstrut{\rule[-0.9ex]{0pt}{0pt}}   % = `bottom' strut

% My commands
\newcommand{\statement}[2] {\textit{\textbf{#1} #2}}
\newcommand{\e} {\varepsilon}
\newcommand{\needpicture} {
    {
        \begin{center}
        \rowcolors{1}{yellow!75}{yellow!75}
        \begin{tabular}{|c|}
            \hline
            Здесь нужна картинка\\
            \hline
        \end{tabular}
        \end{center}
    }
}

\newtheorem{theorem}{Теорема}[section]
\newtheorem{definition}{Определение}[section]
\newtheorem{example}{Пример}[section]
\newtheorem{problem}{Задача}[section]
\newtheorem{lemma}{Лемма}[section]

%Author tags
\newcommand{\groth} {Grothendieck A.}

%%%%%%%%%%%%%%%%%
%     Title     %
%%%%%%%%%%%%%%%%%
\title{Алгосы, часть ii}
\author{Денис Осипов, Иван Ермошин, Alexander Grothendieck}
\date{\today}

\begin{document}

\maketitle

%\lstset{%
%emph={%  
%    if, for, parallel, else, function, to, while, such, that, and, is%
%    },
%    emphstyle={\bfseries},%
%}%

\section*{Введение}

Этот проект~-- коллективный \textbf{конспект} по второй части курса <<Математические основы алгоритмов>>, впервые прочитанного первокурсникам МКН СПбГУ в первой половине ii семестра 2020 года Эдуардом Алексеевичем Гиршем.

Актуальные исходники: \url{https://www.overleaf.com/read/hnbkrkyknbpk} и \url{https://github.com/gogochushij/algosi-hirsch}

Если вы хотите \textbf{принять участие} в написании билетов, или же
\textbf{сообщить об ошибке}, напишите \url{http://vk.com/gogochushij}. Предполагается, что каждый автор напишет около 4 билетов, но мы будем рады любой посильной помощи. \href{https://docs.google.com/spreadsheets/d/17MKhLVzCyYvEKlm6W5Bb-6uUDNyv0QLmdvuh4N6JfXI/edit?usp=sharing}{\texttt{Здесь}} можно посмотреть, с какими билетами вы можете помочь проекту.

Символом \heart~обозначены разделы, которые не входят в экзамен (еще не всё проставлено).

\tableofcontents \newpage

%%%%%%%%%%%%%%%%%%%%%%%%%%%%%%%%%%%%%%%%%
% Здесь настроить конспект для компиляции
% Чтобы скомпилировать все, что есть, впишите:
% \foreach \n \in {1,...,7, 12, 13, 20, 21} {\input{que\n.tex}} 
% ВМЕСТО строки ниже
\foreach \n in {1,2,3,4,5,6,7,8,9,12,13,14,15,20,21} {\input{que\n.tex}}
%%%%%%%%%%%%%%%%%%%%%%%%%%%%%%%%%%%%%%%%%

% que6.tex ссылается на que5.tex
% que21.tex ссылается на que20.tex

\end{document}
